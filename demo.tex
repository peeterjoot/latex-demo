%
% Copyright � 2022 Peeter Joot.  All Rights Reserved.
% Licenced as described in the file LICENSE under the root directory of this GIT repository.
%
%{
\input{../latex/blogpost.tex}
\renewcommand{\basename}{demo}
%\renewcommand{\dirname}{notes/phy1520/}
\renewcommand{\dirname}{notes/ece1228-electromagnetic-theory/}
%\newcommand{\dateintitle}{}
%\newcommand{\keywords}{}

\input{../latex/peeter_prologue_print2.tex}

\usepackage{peeters_layout_exercise}
\usepackage{peeters_braket}
\usepackage{peeters_figures}
\usepackage{siunitx}
\usepackage{verbatim}
%\usepackage{mhchem} % \ce{}
%\usepackage{macros_bm} % \bcM
%\usepackage{macros_qed} % \qedmarker
%\usepackage{txfonts} % \ointclockwise

\beginArtNoToc
\generatetitle{Demo}

Given a two variable problem
\begin{dmath}\label{eqn:demo:20}
   x \Ba + y \Bb = \Bc,
\end{dmath}
we can wedge with \( \Ba \) or \( \Bb \) to remove dependencies on the other.  Example, wedging with \( \Ba \)
\begin{dmath}\label{eqn:demo:40}
   \lr{ \cancel{x \Ba} + y \Bb } \wedge \Ba = \Bc \wedge \Ba,
\end{dmath}
so we are left with
\begin{dmath}\label{eqn:demo:60}
   y \lr{ \Bb \wedge \Ba } = \Bc \wedge \Ba.
\end{dmath}
Now we have the solution for \( y \) if it exists
\begin{dmath}\label{eqn:demo:80}
   y = \frac{ \Bc \wedge \Ba }{ \Bb \wedge \Ba }.
\end{dmath}
Similarly, we can wedge \cref{eqn:demo:20} with \( \Bb \), and get
\begin{dmath}\label{eqn:demo:100}
   x \lr{ \Ba \wedge \Bb } = \Bc \wedge \Bb,
\end{dmath}
We get
\begin{dmath}\label{eqn:demo:120}
   x = \frac{ \Bc \wedge \Bb }{ \Ba \wedge \Bb }.
\end{dmath}


%}
%\EndArticle
\EndNoBibArticle
